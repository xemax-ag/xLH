%! Author = tru
%! Date = 11.06.2025

% Define lernplattform name
\newcommand{\xlhPlattformID}{xLH-io-base}

\documentclass[10pt]{datasheet}
% https://github.com/PetteriAimonen/latex-datasheet-template

% Input encoding and typographical rules for English language
\usepackage[utf8]{inputenc}
\usepackage[ngerman]{babel}
\usepackage[english]{isodate}
\usepackage{lmodern}

% tikz is used to draw images in this example, but you can
% also use \includegraphics{}.
\usepackage{tikz}
\usepackage{pgfplots}
\usepackage{circuitikz}
\usetikzlibrary{calc}

% These define global texts that are used in headers and titles.
\title{Beschreibung \xlhPlattformID}
\author{xemax ag}
\date{Juni 2025}
\revision{Revision 1}
%\companylogo{\Huge $\nabla$ Fancy Company}
\companylogo{\includegraphics[width=5cm]{graphics/xemax_logo}}

\begin{document}
\maketitle


%\section{Beschreibung}

Der \xlhPlattformID erlaubt die Integration von bestehender Infrastruktur in einer kompakten Form.

Weitere Unterlagen (Elektronikschema, Step-Datei Gehäuse, ...) sind via
\href{https://github.com/xemax-ag/xLH/}{GitHub \xlhPlattformID}\ verfügbar.

\section{Features}

\begin{itemize}
    \item Microcontroller M5-Stack Atom Lite
    \item 8 digitale Eingänge 24VDC
    \item 8 digitale Ausgänge 24VDC
    \item 2 analoge Eingänge 0-10VDC
    \item 2 analoge Ausgänge 0-10VDC
    \item Spannungsversorgung für xLH-lx-base integriert
    \item kompakte Umsetzung
\end{itemize}

\section{Software}

\begin{itemize}
    \item ??
\end{itemize}

\section{Interface}

\begin{itemize}
    \item ??
\end{itemize}

% Switch to next column
\vfill\break

\begin{figure}[h]
    \centering
    \includegraphics[width=0.4\textwidth]{graphics/\xlhPlattformID}
\end{figure}

\section{Mechanik}

\begin{itemize}
    \item Formfaktor ?? mm x ?? mm x ?? mm
    \item C-DIN-Schienenhalterung
\end{itemize}

\section{Anwendungen}

\begin{itemize}
    \item Digitaltechnik
    \item Automation
    \item Programmierung
    \item Netzwerktechnik
    \item Technische Grundlagen
    \item Elektronik
\end{itemize}

% Switch to next column
\vfill\break


% For wide tables, a single column layout is better. It can be switched
% page-by-page.
\onecolumn

\section{Optionen}
Der \xlhPlattformID\ kann in verschiedenen Ausführungen bestellt werden.
Eine detaillierte Auflistung ist in Tabelle~\ref{tab:optionen} ersichtlich.

\begin{table}[h]
\begin{threeparttable}
\caption{Optionen \xlhPlattformID}
    \begin{tabularx}{\textwidth}{l | c | l | l | X}

        \thickhline
        \textbf{Pos.} & \textbf{Anzahl} & \textbf{Art-Nr} & \textbf{Beschreibung} & \textbf{Ausführung} \\
        \hline
        1 & 1 & xlh-io-base-pcb-smd & Elektronik PCB mit SMD-Bestückung & \multirow{4}{*}{Bausatz\tnote{1}}  \\
        2 & 1 & xlh-io-base-tht & Elektronik THT Bauteile lose verpackt &  \\
        3 & 1 & xlh-m5s-atom-lite & Microcontroller M5-Stack Atom Lite & \\
        4 & 1 & xlh-io-base-3d-sp & Gehäuse 3D gedruckt inklusive Kleinmaterial & \\
        \hline
        5 & 1 & xlh-io-base-assembly & Zusammenbau Elektronik und Gehäuse & montiert \\
        \hline
        6 & 1 & xlh-c-mount-sp & Adapter C-Schiene inklusive Kleinmaterial & \multirow{4}{*}{Optionen\tnote{2}} \\
        7 & 1 & xlh-ps-24v-65w & Tischnetzgerät 24VDC/2.7A inklusive Kabel 1m &  \\
        8 & 1 & xlh-usb-c-a & USB-C Kabel (Typ A Stecker) &  \\
        9 & 1 & xlh-usb-iso & USB-Adapter galvanische Trennung (Typ A) ADUM3160 &  \\
        \thickhline
    \end{tabularx}
\begin{tablenotes}
\item[1]{Standardmässige Ausführung}
\item[2]{können unabhängig voneinander kombiniert werden}
\end{tablenotes}
\label{tab:optionen}
\end{threeparttable}
\end{table}

\section{Applikationen}
In Tabelle~\ref{tab:applikationen} sind die verfügbaren Applikationen aufgelistet.

\begin{table}[h]
\begin{threeparttable}
\caption{Applikationen \xlhPlattformID}
    \begin{tabularx}{\textwidth}{l | l | l }
        \thickhline
        \textbf{Pos.} & \textbf{Art-Nr} & \textbf{Beschreibung} \\
        \hline
        1 & xlh-io-base-firmware & Firmware für den Microcontroller \\
        2 & ? & ?  \\
        \thickhline
    \end{tabularx}

\label{tab:applikationen}
\end{threeparttable}
\end{table}

\section{Voraussetzungen}
In Tabelle~\ref{tab:voraussetzungen} sind die für den Betrieb benötigten Komponenten aufgelistet.

\begin{table}[h]
\begin{threeparttable}
\caption{Voraussetzungen \xlhPlattformID}
    \begin{tabularx}{\textwidth}{l | l | l }
        \thickhline
        \textbf{Pos.} & \textbf{Art-Nr} & \textbf{Beschreibung} \\
        \hline
        1 & xlh-lx-base & CPU 1GHz quad-core, 64-bit ARM Cortex-A53, 512MB RAM \\
        2 & xlh-lx-power & CPU 1.8GHz quad-core, 64-bit ARM Cortex-A72, 8GB RAM \\
        \thickhline
    \end{tabularx}

\label{tab:voraussetzungen}
\end{threeparttable}
\end{table}

\end{document}