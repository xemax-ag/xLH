%! Author = tru
%! Date = 11.06.2025

%\section{Beschreibung}

Der \textbf{\lernplattformname}\ stellt das Einsteigermodell für die xLH-Familie dar.
Ein Linux Embeddedrechner für den kostenoptimierten Einsatz im Bereiche SPS-Technologie in der beruflichen Grundbildung.
Sinnvollerweise wird diese Komponente als persönliches Material verwendet.

Weitere Unterlagen (Elekronikschema, Step-Datei Gehäuse, Linux-Image, Applikationen, ...) sind via
\href{https://github.com/xemax-ag/xLH/}{GitHub \lernplattformname}\ verfügbar.

\section{Features}

\begin{itemize}
    \item CPU 1GHz quad-core, 64-bit ARM Cortex-A53, 512MB RAM
    \item Linux Debian Bookworm Distribution
    \item 4 Taster, 4 Schalter, 7 LEDs
    \item 128 x 64 OLED
    \item RGB Matrix 8x8
    \item USB-C Speisung (ohne PD Funktionalität)
\end{itemize}

\section{Software}

\begin{itemize}
    \item SPS \href{https://de.wikipedia.org/wiki/EN_61131}{IEC61131-3} Stack von \href{https://www.codesys.com/}{Codesys}
    \item \href{https://www.python.org/}{Python}
    \item \href{https://jupyter.org/try-jupyter/lab/}{Jupyter Lite}
\end{itemize}

\section{Interface}

\begin{itemize}
    \item WiFi, Bluethooth, RJ45 Ethernet via Adapter
    \item CAN-Bus
    \item Erweiterungsstecker Grove
\end{itemize}

% Switch to next column
\vfill\break

\begin{figure}[h]
    \centering
    \includegraphics[width=0.25\textwidth]{graphics/xLH_base.png}
\end{figure}

\section{Mechanik}

\begin{itemize}
    \item Formfaktor 135 mm x 60 mm x 45 mm
    \item Deckel mit Magnethalterung
    \item C-DIN-Schienenhalterung
\end{itemize}

\section{Anwendungen}

\begin{itemize}
    \item Digitaltechnik
    \item Automation
    \item Programmierung
    \item Netzwerktechnik
    \item Technische Grundlagen
    \item Elektronik
\end{itemize}

% Switch to next column
\vfill\break


% For wide tables, a single column layout is better. It can be switched
% page-by-page.
\onecolumn

\section{Optionen}
Der \lernplattformname\ kann in verschiedenen Ausführungen bestellt werden.
Eine detaillierte Auflistung ist in Tabelle~\ref{tab:optionen} ersichtlich.

\begin{table}[h]
\begin{threeparttable}
\caption{Optionen \lernplattformname}
    \begin{tabularx}{\textwidth}{l | c | l | l | X}

        \thickhline
        \textbf{Pos.} & \textbf{Anzahl} & \textbf{Art-Nr} & \textbf{Beschreibung} & \textbf{Ausführung} \\
        \hline
        1 & 1 & xlh-base-pcb-smd & Elektronik PCB mit SMD-Bestückung & \multirow{5}{*}{Bausatz\tnote{1}}  \\
        2 & 1 & xlh-base-tht & Elektronik THT Bauteile lose verpackt &  \\
        3 & 1 & xlh-base-rpiz2w & Raspberry Pi Zero 2 W & \\
        4 & 1 & xlh-ucsd-32 & Micro-SD Karte 32GB & \\
        5 & 1 & xlh-base-3d-sp & Gehäuse 3D gedruckt inklusive Kleinmaterial & \\
        \hline
        6 & 1 & xlh-base-assembly & Montagearbeiten Zusammenbau Elektronik und Gehäuse & montiert \\
        \hline
        7 & 1 & xlh-c-mount-sp & Adapter C-Schiene inklusive Kleinmaterial & \multirow{4}{*}{Optionen\tnote{2}} \\
        8 & 1 & xlh-card-reader & Micro-SD Card Reader für PC (Download Image) &  \\
        9 & 1 & xlh-usb-uc-a & Micro-USB Kabel (Typ A Stecker) &  \\
        10 & 1 & xlh-usb-c-a & USB-C Kabel (Typ A Stecker) &  \\
        \thickhline
    \end{tabularx}
\begin{tablenotes}
\item[1]{Standardmässige Ausführung}
\item[2]{können unabhängig voneinander kombiniert werden}
\end{tablenotes}
\label{tab:optionen}
\end{threeparttable}
\end{table}


\section{Applikationen}
In Tabelle~\ref{tab:applikationen} sind die verfügbaren Applikationen aufgelistet.

\begin{table}[h]
\begin{threeparttable}
\caption{Applikationen \lernplattformname}
    \begin{tabularx}{\textwidth}{l | l | l }
        \thickhline
        \textbf{Pos.} & \textbf{Art-Nr} & \textbf{Beschreibung} \\
        \hline
        1 & xlh-base-app-iotest & Start-Applikation mit IO-Test für die Inbetriebnahme \\
        2 & ? & ?  \\
        \thickhline
    \end{tabularx}

\label{tab:applikationen}
\end{threeparttable}
\end{table}

