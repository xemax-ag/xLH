%! Author = tru
%! Date = 11.06.2025

xEduOrg ist eine innovative Softwarelösung, die speziell für die Entwicklung von
Bildungsprodukten konzipiert wurde.
Mit xEduOrg können Berufsverbände und Berufsentwickler Handlungskompetenzbereiche, Handlungskompetenzen,
Leistungskriterien und Lernziele modellieren, verknüpfen und visualisieren.
Ebenso ist es möglich die Leistungskriterien mittels Lernfeldern
(Lernfeldbereichen und Kompetenznachweisen) und Lernzielen zu operationalisieren.
Das Tool bietet eine einzigartige Möglichkeit, Bildungsprozesse zu strukturieren und zu optimieren, indem es die
komplexen Zusammenhänge klar und verständlich darstellt.

Es eignet sich sowohl für Bildungsprodukte aus der beruflichen Grundbildung als auch für Produkte aus der höheren
Berufsbildung.

Sie besteht als Basis aus einer Datenbank mit Bedienungsoberfläche als
\IfStrEq{\projectname}{futuremem}{\href{https://prozesse.xemax.ch/fmi/webd/FUTUREMEM-LFE}{Web-Applikation}.}{}
\IfStrEq{\projectname}{vssm-frecem}{\href{https://prozesse.xemax.ch/fmi/webd/VSSM-FRECEM}{Web-Applikation}.}{}
Für die reine Betrachtung und Verbreitung der Informationen steht eine
\IfStrEq{\projectname}{futuremem}{\href{https://skills.futuremem.swiss/de/}{Webseite} zur Verfügung
    (siehe Abbildung~\ref{fig:topologie-xeduorg}).}{}
\IfStrEq{\projectname}{vssm-frecem}{\href{https://vssm-frecem-skills-basic.xemax.ch/de/}{Webseite Grundbildung} und
    \href{https://vssm-frecem-skills-advanced.xemax.ch/de/}{Webseite Höhere Berufsbildung} zur Verfügung
    (siehe Abbildung~\ref{fig:topologie-xeduorg}).}{}

\begin{figure}[h!]
    \centering
    \includegraphics[width=0.5\textwidth]{graphics/topologie_xeduorg}
    \caption{Topologie xEduOrg}
    \label{fig:topologie-xeduorg}
\end{figure}

Erweiterte Funktionalitäten werden auf einen zusätzlichen Server ausgelagert.

\linebreak
Mehrwert:

\begin{itemize}
    \item Strukturierte Datenhaltung als Basis für weitere Prozessschritte/Anwendungen
    \item Analysen und Bereinigung der bestehenden Daten (Redundanzen, Verdichtung, …)
    \item Qualitäts- und Effizienzsteigerung durch Automation
    \item zeitgemässer Auftritt gegenüber Dritten
\end{itemize}

